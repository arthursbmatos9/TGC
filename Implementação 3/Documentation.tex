\documentclass[12pt]{article}
\usepackage[left=25mm,right=25mm,top=25mm,bottom=25mm,paper=a4paper]{geometry}
\usepackage{ragged2e}
\usepackage{titlesec}
\usepackage[T1]{fontenc}
\usepackage{babel}
\usepackage{amsmath}

% Define o formato para os títulos das seções
\titleformat{\section}
  {\normalfont\bfseries\fontsize{14}{14}\selectfont\raggedright} % Fonte e estilo
  {\thesection}{1em} % Número da seção
  {} % Antes do título
  [] % Após o título


% %%%%%%%%%%%%%%%%%%%%%%%%%%%%%%%%%%%%%%%%%%%%%%%%%%%%%%%%%%%%%%
% CABEÇALHO
\title{\textbf{Documentação Implementação 3}} % titulo

\author{
  Arthur de Sá Braz de Matos\\
}
\date{}

% %%%%%%%%%%%%%%%%%%%%%%%%%%%%%%%%%%%%%%%%%%%%%%%%%%%%%%%%%%%%%%

\begin{document}
\maketitle
Esta é a documentação da implementação 3 da disciplina de Teoria dos Grafos e Computabilidade. Os códigos foram desenvolvidos em C++ e tem como objetivo implementar os algoritmos de Dijkstra, MinMax e MaxMin para grafos conexos, direcionados e com pesos positivos, bem como possíveis aplicações no mundo real.


\section{Dijkstra}

\subsection{Estruturas de Dados}

O algoritmo utiliza as seguintes estruturas de dados:
\begin{itemize}
    \item \textbf{Vertices e Visitados}: Duas listas:
    \begin{itemize}
        \item \texttt{vertices}: Contém todos os vértices do grafo.
        \item \texttt{visitados}: Armazena os vértices que já foram processados pelo algoritmo.
    \end{itemize}
    
    \item \textbf{Distâncias}: Um vetor \texttt{distancias} guarda a menor distância encontrada até o momento de cada vértice em relação ao vértice de origem. Inicialmente, todas as distâncias são definidas como \(\infty\), exceto a distância do vértice inicial, que é 0.
    
    \item \textbf{Matriz de Adjacência}: Uma matriz bidimensional \texttt{matriz} guarda os pesos das arestas que conectam os vértices. O valor na posição \texttt{matriz[i][j]} representa o peso da aresta do vértice \(i\) para o vértice \(j\).
\end{itemize}

\subsection{Funcionamento do Algoritmo}

O método principal é \texttt{shortestPath(int u)}, que executa os seguintes passos:

\begin{enumerate}
    \item \textbf{Inicialização}: A distância do vértice de origem (\(u\)) para si mesmo é definida como 0.
    
    \item \textbf{Laço Principal}: Enquanto houver vértices não visitados, o algoritmo seleciona aquele com a menor distância acumulada utilizando a função \texttt{menorDistancia()}.
    
    \item \textbf{Atualização de Distâncias}: Para o vértice com menor distância acumulada, o algoritmo atualiza as distâncias dos seus vértices vizinhos. A nova distância é calculada como:
    \begin{align*}
    \texttt{distancias[v]} = \min \big(&\texttt{distancias[v]}, \\
    &\texttt{distancias[vMenorDistancia]} + \texttt{matriz[vMenorDistancia][v]} \big)
\end{align*}
    Dessa forma, pegamos o menor valor entre o atual e o valor do vértice de saída somado com o tamanho da aresta que os conecta.
    Sendo \texttt{vMenorDistancia} o vértice com a menor distância acumulada, encontrado anteriormente.
    
    \item \textbf{Marcação de Visitados}: Após atualizar as distâncias dos vizinhos, o vértice é marcado como visitado e o processo se repete até que todos os vértices sejam visitados.
    
    \item \textbf{Retorno das Distâncias}: Ao final, o vetor \texttt{distancias} é retornado, contendo a menor distância de todos os vértices em relação ao vértice de origem.
\end{enumerate}

\subsection{Função \texttt{menorDistancia()}}

A função \texttt{menorDistancia()} percorre os vértices não visitados e retorna aquele que possui a menor distância acumulada. Se todos os vértices foram visitados ou não há mais caminhos possíveis, o algoritmo termina.

\subsection{Exemplo de Uso}

O arquivo de entrada \texttt{graph1.graph} contém a definição dos vértices e suas arestas com os respectivos pesos. O usuário escolhe um vértice inicial e um vértice final. O programa então calcula e exibe a menor distância entre esses dois vértices utilizando o algoritmo de Dijkstra.

\subsection{Resultado}

Ao final da execução, o programa imprime a menor distância entre os vértices de origem e destino escolhidos, ou exibe uma mensagem de erro caso não exista um caminho possível. Além disso, o vetor de distâncias completo é exibido.

\section{Algoritmo MinMax}

O algoritmo MinMax é utilizado para encontrar o caminho que minimiza o peso máximo entre os vértices de um grafo ponderado. Esse algoritmo é uma variante do problema de menor caminho, mas em vez de buscar minimizar a soma dos pesos das arestas ao longo de um caminho, ele visa minimizar o valor máximo entre essas arestas.

Abaixo está a explicação detalhada do algoritmo:

\subsection{Descrição}

Dado um grafo ponderado representado por uma matriz de adjacência \( \texttt{matriz} \), onde o valor \( \texttt{matriz}[i][j] \) representa o peso da aresta que vai do vértice \( i \) para o vértice \( j \), o algoritmo MinMax realiza os seguintes passos:

\begin{enumerate}
    \item Inicialmente, todos os vértices têm seus pesos definidos como \( \infty \), exceto o vértice de origem \( u \), cujo peso é \( -\infty \). 
    \item A cada iteração, escolhe-se o vértice com o menor peso ainda não visitado e marca-o como visitado.
    \item Para cada vizinho \( v \) desse vértice, calcula-se o maior valor entre o peso atual do vértice \( u \) e o peso da aresta que liga \( u \) a \( v \). O valor obtido é comparado com o peso do vértice \( v \), e o menor valor entre eles é atualizado no vetor de pesos.
    \item O processo se repete até que todos os vértices tenham sido visitados ou não haja mais vértices acessíveis.
\end{enumerate}

O objetivo final do algoritmo é encontrar o menor valor máximo de um caminho entre o vértice inicial e os outros vértices do grafo. Esse valor é armazenado na variável \( \texttt{valorMinMax} \), que é atualizado a cada iteração.

\subsection{Equações}

A equação utilizada no algoritmo é a seguinte:

\[
\texttt{valorMinMax} = \min \left( \max(\texttt{pesos}[vMenorPeso],  \texttt{matriz}[vMenorPeso][v]),  \texttt{pesos}[v] \right)
\]

\noindent Essa equação compara o peso do caminho entre os vértices e atualiza o peso do vértice vizinho, garantindo que o peso máximo ao longo do caminho seja minimizado.

\section{Algoritmo MaxMin}

O algoritmo MaxMin é utilizado para encontrar o caminho que maximiza o menor peso entre os vértices de um grafo ponderado. Diferentemente do algoritmo MinMax, onde buscamos minimizar o valor máximo das arestas, o MaxMin busca maximizar o valor mínimo encontrado ao longo de um caminho.

\subsection{Descrição}

Dado um grafo ponderado representado por uma matriz de adjacência \( \texttt{matriz} \), onde o valor \( \texttt{matriz}[i][j] \) representa o peso da aresta que vai do vértice \( i \) para o vértice \( j \), o algoritmo MaxMin realiza os seguintes passos:

\begin{enumerate}
    \item Inicialmente, todos os vértices têm seus pesos definidos como \( -\infty \), exceto o vértice de origem \( u \), cujo peso é \( \infty \).
    \item Para cada iteração, escolhe-se o vértice com o maior peso ainda não visitado, com base no número de arestas não percorridas. Caso esse vértice não tenha mais arestas adjacentes, ele é marcado como visitado.
    \item Para cada vizinho \( v \), calcula-se o menor valor entre o peso atual do vértice \( u \) e o peso da aresta que liga \( u \) a \( v \). O valor obtido é comparado com o peso do vértice \( v \), e o maior valor entre eles é atualizado no vetor de pesos.
    \item O processo se repete até que todos os vértices tenham sido visitados ou não haja mais vértices acessíveis.
\end{enumerate}

O objetivo do algoritmo é maximizar o valor mínimo de um caminho entre o vértice inicial e os outros vértices do grafo. Esse valor é armazenado na variável \( \texttt{valorMaxMin} \), que é atualizado durante cada iteração.

\subsection{Equações}

A equação fundamental utilizada no algoritmo MaxMin é a seguinte:

\[
\texttt{valorMaxMin} = \max \left( \min(\texttt{pesos}[vMaiorPeso], \texttt{matriz}[vMaiorPeso][v]), \texttt{pesos}[v] \right)
\]

\noindent Essa equação garante que o peso de cada caminho seja atualizado considerando o mínimo entre os pesos da aresta atual e o maior valor mínimo calculado até o momento.

\section{Aplicações reais}
\subsection{Dijkstra}
\begin{enumerate}
    \item Navegação GPS e Sistemas de Mapas: Softwares como Google Maps e Waze utilizam o Dijkstra para calcular a rota mais curta entre dois locais, levando em consideração a distância, tempo de viagem ou evitar pedágios.
    \item Transporte Público: Sistemas de transporte como trens e ônibus usam o Dijkstra para otimizar rotas, garantindo que passageiros possam chegar ao destino final da forma mais rápida possível, considerando conexões, tempos de espera, etc.
\end{enumerate}

\subsection{MinMax}
\begin{enumerate}
    \item Planejamento Urbano: O MinMax pode ser aplicado em planejamento urbano. Por exemplo, ao construir hospitais, o objetivo pode ser minimizar a distância máxima que os cidadãos terão que percorrer para acessar esse serviço, garantindo que todos estejam relativamente próximos.
    \item Design de Redes Elétricas: Ao projetar uma rede elétrica, é importante minimizar a carga máxima em qualquer ponto da rede para evitar falhas e sobrecargas. O algoritmo MinMax pode ser utilizado para distribuir a carga de maneira eficiente, minimizando o risco de um ponto crítico ser sobrecarregado.
\end{enumerate}

\subsection{MaxMin}
\begin{enumerate}
    \item Design de Redes de Energia: No design de redes elétricas, o algoritmo MaxMin pode ser utilizado para garantir que a menor carga transmitida em qualquer ponto da rede seja maximizada, assegurando que nenhum segmento da rede receba uma carga muito baixa, o que poderia causar instabilidades.
    \item Gerenciamento de Riscos em Investimentos: O algoritmo MaxMin pode ser aplicado na diversificação de carteiras de investimento. A ideia é maximizar o mínimo retorno esperado, garantindo que o pior retorno seja o melhor possível, minimizando os riscos associados a perdas financeiras.
\end{enumerate}

\end{document}