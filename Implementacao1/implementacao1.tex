\documentclass{article}
\usepackage{graphicx} % Required for inserting image

\begin{document}

\begin{center}
    \Large\textbf{Implementação 1} \\
    \large Arthur de Sá Braz de Matos \\
    \normalsize Agosto 2024
\end{center}

\vspace{1cm} % Adiciona um espaço de 1 cm

\section{O que foi feito?}
Foram desenvolvidos quatro códigos, cada um correspondendo a uma estrutura de dados distinta: lista, pilha, fila e matriz. As características de cada estrutura foram respeitadas, conforme detalhado abaixo:

\begin{itemize}
    \item \textbf{Pilha}: A inserção e a remoção de elementos ocorrem no topo (início) da estrutura.
    \item \textbf{Fila}: A inserção de elementos ocorre no final da fila, enquanto a remoção é feita no início.
    \item \textbf{Lista}: A lista permite a inserção e remoção de elementos em qualquer posição (início, fim ou intermediária). Optou-se por realizar essas operações no início da lista.
    \item \textbf{Matriz}: Uma matriz foi criada para armazenar os elementos de forma bidimensional.
\end{itemize}

Além disso, foram implementadas funções para verificar a presença de elementos nas estruturas de dados. Essas funções percorrem sequencialmente as estruturas em busca do elemento desejado. Caso o elemento seja encontrado, a função retorna imediatamente, evitando processamento desnecessário. Se o elemento não estiver presente, a função indica sua ausência.
\end{document}
